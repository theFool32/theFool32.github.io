%%%%%%%%%%%%%%%%%%%%%%%%%%%%%%%%%%%%%%%%%
% Medium Length Graduate Curriculum Vitae
% LaTeX Template
% Version 1.1 (9/12/12)
%
% This template has been downloaded from:
% http://www.LaTeXTemplates.com
%
% Original author:
% Rensselaer Polytechnic Institute (http://www.rpi.edu/dept/arc/training/latex/resumes/)
%
% Important note:
% This template requires the res.cls file to be in the same directory as the
% .tex file. The res.cls file provides the resume style used for structuring the
% document.
%
%%%%%%%%%%%%%%%%%%%%%%%%%%%%%%%%%%%%%%%%%

%----------------------------------------------------------------------------------------
%	PACKAGES AND OTHER DOCUMENT CONFIGURATIONS
%----------------------------------------------------------------------------------------

\documentclass[margin, 10pt]{res} % Use the res.cls style, the font size can be changed to 11pt or 12pt here

\usepackage{helvet} % Default font is the helvetica postscript font
%\usepackage{newcent} % To change the default font to the new century schoolbook postscript font uncomment this line and comment the one above

\setlength{\textwidth}{5.1in} % Text width of the document

\begin{document}

%----------------------------------------------------------------------------------------
%	NAME AND ADDRESS SECTION
%----------------------------------------------------------------------------------------

\moveleft.5\hoffset\centerline{\large\bf Jie Li} % Your name at the top

\moveleft\hoffset\vbox{\hrule width\resumewidth height 1pt}\smallskip % Horizontal line after name; adjust line thickness by changing the '1pt'

\moveleft.5\hoffset\centerline{Third-year Ph.D student $\cdot$ Xiamen University} % Your address
\moveleft.5\hoffset\centerline{Address: Room 701, Administration Building \#B, Haiyun Park, Xiamen University, 361005} % Your address
\moveleft.5\hoffset\centerline{E-mail: lijie.32@outlook.com}
\moveleft.5\hoffset\centerline{Person Homepage: https://m0re.fun}

%----------------------------------------------------------------------------------------

\begin{resume}

%----------------------------------------------------------------------------------------
%	OBJECTIVE SECTION
%----------------------------------------------------------------------------------------

\section{RESEARCH INTEREST}
Machine Learn and Computer Vision. \\
Adversarial examples and robust deep learning.


%----------------------------------------------------------------------------------------
%	EDUCATION SECTION
%----------------------------------------------------------------------------------------

\section{EDUCATION}

{\sl M.S-Ph.D. candidate of Artificial Intelligence} \hfill 2019.09 - PRESENT \\
Xiamen University (XMU), Xiamen, China \\
Advisor: Rongrong Ji

{\sl Master of Artificial Intelligence} \hfill 2017.09 - 2019.06 \\
Xiamen University (XMU), Xiamen, China \\
Advisor: Rongrong Ji

{\sl Bachelor of Computer Science} \hfill 2013.09 - 2017.06 \\
Xiamen University (XMU), Xiamen, China

\section{PUBLICATION}

Yixu Wang, \textbf{Jie Li}, Hong Liu, Yan Wang, Yongjian Wu, Feiyue Huang, Rongrong Ji. \textit{Black-Box Dissector: Towards Erasing-based Hard-Label Model Stealing Attack}.
In European Conference on Computer Vision (ECCV 2022).

Shuman Fang, \textbf{Jie Li}, Xianming Lin, Rongrong Ji. \textit{Learning to Learn Transferable Attack}.
In Proceedings of the AAAI Conference on Artificial Intelligence (AAAI 2022).

\textbf{Jie Li}, Rongrong Ji, Peixian Chen, Baochang Zhang, Xiaopeng Hong, Ruixin Zhang, Shaoxin Li, Jilin Li, Feiyue Huang, Yongjian Wu. \textit{Aha! Adaptive History-driven Attack for Decision-based Black-box Models}.
In Proceedings of the International Conference on Computer Vision 2021 (ICCV 2021).

\textbf{Jie Li}, Rongrong Ji, Hong Liu, Jianzhuang Liu, Bineng Zhong, Cheng Deng, Qi Tian. \textit{Projection \& Probability-Driven Black-Box Attack}.
In Proceedings of the Conference on ComputerVision and Pattern Recognition 2020 (CVPR 2020).

Hong Liu, Rongrong Ji, \textbf{Jie Li}, Baochang Zhang, Yue Gao, Yongjian Wu, Feiyue Huang. \textit{Universal Adversarial Perturbation via Prior Driven Uncertainty Approximation}.
In Proceedings of the International Conference on Computer Vision 2019 (ICCV 2019). (Oral).

\textbf{Jie Li}, Rongrong Ji, Hong Liu, Xiaopeng Hong, Yue Gao, Qi Tian. \textit{Universal Perturbation Attack Against Image Retrieval}.
In Proceedings of the International Conference on Computer Vision 2019 (ICCV 2019).

Hong Liu, \textbf{Jie Li}, Rongrong Ji, and Yongjian Wu. \textit{Learning Neural Bag-of-Matrix-Summarization with Riemannian Network}.
In Proceedings of the AAAI Conference on Artificial Intelligence (AAAI 2019).

Xianming Lin, \textbf{Jie Li}, Hualin Zeng, Rongrong Ji. \textit{Font generation based on least squares conditional generative adversarial nets}.
Multimedia Tools and Applications, 2019.

\section{RESEARCH EXPERIENCE}
Research Intern \hfill 2020.06 - 2020.08 \\
Youtu Lab, Tencent Technology (Shanghai) CO.,Ltd, China
\begin{itemize}
    \item Decision-based Black-box Adversarial Attack (Published at ICCV'21)
\end{itemize}

\section{AWARDS}
Xiamen University Scholarship \hfill 2020

\section{ACTIVITIES}
Reviewer: TIP, CVPR, ICCV, ECCV, AAAI, IJCAI, ACM MM, ACCV, WACV, MM Asia.

\end{resume}
\end{document}

%%% Local Variables:
%%% mode: latex
%%% TeX-master: t
%%% End:
